% Version 2.1 April 2023
% See section 11 of the User Manual for version history
%
%%%%%%%%%%%%%%%%%%%%%%%%%%%%%%%%%%%%%%%%%%%%%%%%%%%%%%%%%%%%%%%%%%%%%%
%%                                                                 %%
%% Please do not use \input{...} to include other tex files.       %%
%% Submit your LaTeX manuscript as one .tex document.              %%
%%                                                                 %%
%% All additional figures and files should be attached             %%
%% separately and not embedded in the \TeX\ document itself.       %%
%%                                                                 %%
%%%%%%%%%%%%%%%%%%%%%%%%%%%%%%%%%%%%%%%%%%%%%%%%%%%%%%%%%%%%%%%%%%%%%

%%\documentclass[referee,sn-basic]{sn-jnl}% referee option is meant for double line spacing

%%=======================================================%%
%% to print line numbers in the margin use lineno option %%
%%=======================================================%%

%%\documentclass[lineno,sn-basic]{sn-jnl}% Basic Springer Nature Reference Style/Chemistry Reference Style

%%======================================================%%
%% to compile with pdflatex/xelatex use pdflatex option %%
%%======================================================%%

%%\documentclass[pdflatex,sn-basic]{sn-jnl}% Basic Springer Nature Reference Style/Chemistry Reference Style


%%Note: the following reference styles support Namedate and Numbered referencing. By default the style follows the most common style. To switch between the options you can add or remove �Numbered� in the optional parenthesis. 
%%The option is available for: sn-basic.bst, sn-vancouver.bst, sn-chicago.bst, sn-mathphys.bst. %  
 
%%\documentclass[sn-nature]{sn-jnl}% Style for submissions to Nature Portfolio journals
%%\documentclass[sn-basic]{sn-jnl}% Basic Springer Nature Reference Style/Chemistry Reference Style
%%\documentclass[sn-mathphys,Numbered]{sn-jnl}% Math and Physical Sciences Reference Style
%%\documentclass[sn-aps]{sn-jnl}% American Physical Society (APS) Reference Style
\documentclass[sn-vancouver,Numbered]{sn-jnl}% Vancouver Reference Style
%%\documentclass[sn-apa]{sn-jnl}% APA Reference Style 
%%\documentclass[sn-chicago]{sn-jnl}% Chicago-based Humanities Reference Style
%%\documentclass[default]{sn-jnl}% Default
%%\documentclass[default,iicol]{sn-jnl}% Default with double column layout

%%%% Standard Packages
%%<additional latex packages if required can be included here>

\usepackage{graphicx}%
\usepackage{multirow}%
\usepackage{amsmath,amssymb,amsfonts}%
\usepackage{amsthm}%
\usepackage{mathrsfs}%
\usepackage[title]{appendix}%
\usepackage{xcolor}%
\usepackage{textcomp}%
\usepackage{manyfoot}%
\usepackage{booktabs}%
\usepackage{algorithm}%
\usepackage{algorithmicx}%
\usepackage{algpseudocode}%
\usepackage{listings}%
\usepackage{lstautogobble}%
%%%%

\lstset{
    texcl=true,
    basicstyle=\small\sf,
    commentstyle=\small\rm,
    mathescape=true,
    escapeinside={(*}{*)},
    breaklines=true,
    breakindent=0.5pt
}

%%%%%=============================================================================%%%%
%%%%  Remarks: This template is provided to aid authors with the preparation
%%%%  of original research articles intended for submission to journals published 
%%%%  by Springer Nature. The guidance has been prepared in partnership with 
%%%%  production teams to conform to Springer Nature technical requirements. 
%%%%  Editorial and presentation requirements differ among journal portfolios and 
%%%%  research disciplines. You may find sections in this template are irrelevant 
%%%%  to your work and are empowered to omit any such section if allowed by the 
%%%%  journal you intend to submit to. The submission guidelines and policies 
%%%%  of the journal take precedence. A detailed User Manual is available in the 
%%%%  template package for technical guidance.
%%%%%=============================================================================%%%%

%\jyear{2021}%

%% as per the requirement new theorem styles can be included as shown below
%%\theoremstyle{thmstyleone}%
%%\newtheorem{theorem}{Theorem}%  meant for continuous numbers
%%\newtheorem{theorem}{Theorem}[section]% meant for sectionwise numbers
%% optional argument [theorem] produces theorem numbering sequence instead of independent numbers for Proposition
%%\newtheorem{proposition}[theorem]{Proposition}% 
%%\newtheorem{proposition}{Proposition}% to get separate numbers for theorem and proposition etc.

%%\theoremstyle{thmstyletwo}%
%%\newtheorem{example}{Example}%
%%\newtheorem{remark}{Remark}%

%%\theoremstyle{thmstylethree}%
%%\newtheorem{definition}{Definition}%

\raggedbottom
%%\unnumbered% uncomment this for unnumbered level heads

\begin{document}

\title[Comparing Generative Pre-trained Transformer Models to Physicians for Assigning Causes of Death from Verbal Autopsy: A Case Study of 6939 Deaths in Sierra Leone from 2019-2022]{Comparing Generative Pre-trained Transformer Models to Physicians for Assigning Causes of Death from Verbal Autopsy: A Case Study of 6939 Deaths in Sierra Leone from 2019-2022}

%%=============================================================%%
%% Prefix	-> \pfx{Dr}
%% GivenName	-> \fnm{Joergen W.}
%% Particle	-> \spfx{van der} -> surname prefix
%% FamilyName	-> \sur{Ploeg}
%% Suffix	-> \sfx{IV}
%% NatureName	-> \tanm{Poet Laureate} -> Title after name
%% Degrees	-> \dgr{MSc, PhD}
%% \author*[1,2]{\pfx{Dr} \fnm{Joergen W.} \spfx{van der} \sur{Ploeg} \sfx{IV} \tanm{Poet Laureate} 
%%                 \dgr{MSc, PhD}}\email{iauthor@gmail.com}
%%=============================================================%%

\author*[1]{\fnm{Richard} \sur{Wen}}\email{richard.wen@utoronto.ca}

\author[1,2]{\fnm{Anteneh Tesfaye} \sur{Assalif}}\email{antenehta@gmail.com}

\author[1]{\fnm{Andy Sze-Heng} \sur{Lee}}\email{andylee@cs.toronto.edu}

\author[1]{\fnm{Rajeev} \sur{Kamadod}}\email{rajeevk@kentropy.com}

\author[1]{\fnm{Cheryl} \sur{Chin}}\email{cheryl.chin@unityhealth.to}

%\equalcont{These authors contributed equally to this work.}

\author[1]{\fnm{Asha} \sur{Behdinan}}\email{asha.behdinan@mail.utoronto.ca}

\author[1]{\fnm{Leslie} \sur{Newcombe}}\email{leslie.newcombe@unityhealth.to}

\author[1]{\fnm{Areeba} \sur{Zubair}}\email{areeba.zubair@mail.utoronto.ca}

\author[1]{\fnm{Thomas Kai Sze} \sur{Ng}}\email{kaisze.ng@unityhealth.to}

\author[1]{\fnm{Paijani} \sur{Sheth}}\email{paijani.sheth@mail.utoronto.ca}

\author[1]{\fnm{Patrick} \sur{Brown}}\email{patrick.brown@utoronto.ca}

\author[1]{\fnm{Prabhat} \sur{Jha}}\email{prabhat.jha@utoronto.ca}

\author[2]{\fnm{Rashid} \sur{Ansumana}}\email{rashidansumana@gmail.com}

\affil*[1]{\orgdiv{Centre for Global Health Research, St. Michael's Hospital}, \orgname{Unity Health Toronto and University of Toronto}, \orgaddress{\street{30 Bond St}, \city{Toronto}, \postcode{M5B 1W8}, \state{Ontario}, \country{Canada}}}

\affil[2]{\orgdiv{School of Community Health Sciences}, \orgname{Njala University}, \orgaddress{\city{Bo}, \country{Sierra Leone}}}

%%==================================%%
%% sample for unstructured abstract %%
%%==================================%%

\abstract{\textbf{Background:} Verbal Autopsies (VAs) collect data on deaths and their causes outside of traditional hospital settings to provide more representative counts and Causes of Death (CODs) for reducing premature mortality. Current computer models for COD assignment in VAs perform similar to physicians at the population level, but poorly at the individual level, due to a focus on structured questionnaire data and neglecting free text from the open narratives. Recently, a Generative Pre-trained Transformer (GPT) model called ChatGPT-4 has demonstrated human-level performance on professional and academic exams using free text input. ChatGPT-4 shows promise in mimicking physician behavior for assigning CODs, but to the best of our knowledge, has yet to be tested for assigning CODs using open narratives from VAs.

\textbf{Methods:} 6939 records collected from VA in Sierra Leone from 2019 to 2022 were used to compare four computer models, GPT-3.5, GPT-4, InterVA-5, and InSilicoVA, to physicians for assigning CODs at the population and individual level. Open narratives were used for Chat-3.5/GPT-4 input, while structured questionnaires were used for InterVA-5/InSilicoVA input. All COD assignments were grouped into general COD categories consisting of 19, 10, and 7 categories for the adult, child, and neonatal age groups. Cause Specific Mortality Fraction (CSMF) accuracy and Partial Corrected Concordance (PCCC) were used to compare models to physicians at the population and individual level respectively. Comparisons in CSMF and PCCC to physicians among models were evaluated for all records and by COD, age group, and age ranges.

\textbf{Results:} x.

\textbf{Conclusion:} x.
}

%%================================%%
%% Sample for structured abstract %%
%%================================%%

% \abstract{\textbf{Purpose:} The abstract serves both as a general introduction to the topic and as a brief, non-technical summary of the main results and their implications. The abstract must not include subheadings (unless expressly permitted in the journal's Instructions to Authors), equations or citations. As a guide the abstract should not exceed 200 words. Most journals do not set a hard limit however authors are advised to check the author instructions for the journal they are submitting to.
% 
% \textbf{Methods:} The abstract serves both as a general introduction to the topic and as a brief, non-technical summary of the main results and their implications. The abstract must not include subheadings (unless expressly permitted in the journal's Instructions to Authors), equations or citations. As a guide the abstract should not exceed 200 words. Most journals do not set a hard limit however authors are advised to check the author instructions for the journal they are submitting to.
% 
% \textbf{Results:} The abstract serves both as a general introduction to the topic and as a brief, non-technical summary of the main results and their implications. The abstract must not include subheadings (unless expressly permitted in the journal's Instructions to Authors), equations or citations. As a guide the abstract should not exceed 200 words. Most journals do not set a hard limit however authors are advised to check the author instructions for the journal they are submitting to.
% 
% \textbf{Conclusion:} The abstract serves both as a general introduction to the topic and as a brief, non-technical summary of the main results and their implications. The abstract must not include subheadings (unless expressly permitted in the journal's Instructions to Authors), equations or citations. As a guide the abstract should not exceed 200 words. Most journals do not set a hard limit however authors are advised to check the author instructions for the journal they are submitting to.}

\keywords{Cause of Death, Physician Coding, Verbal Autopsy, GPT}

%%\pacs[JEL Classification]{D8, H51}

%%\pacs[MSC Classification]{35A01, 65L10, 65L12, 65L20, 65L70}

\maketitle

\section{Background}\label{background}

In 2019, 41 million people die prematurely from noncommunicable diseases every year, accounting for 74\% of all deaths globally \cite{worldhealthorganizationNonCommunicableDiseases2019}. Most of these deaths are preventable, but require adequate resource allocation, guided by evidence, to implement effective interventions and policies that target populations at risk \cite{benzigerGlobalBurdenDisease2016}. Thus, reliable counts and diagnoses of deaths enable decision makers to identify populations at risk to save lives and reduce premature deaths worldwide \cite{lawnMillionNeonatalDeaths2010,lassiCommunityBasedIntervention2015,liuExcessMortalityPersons2017,ewigCommunityacquiredPneumoniaEmergency2011}. However, most low-income countries do not have data on deaths or have registered less than half of the deaths in their country, with an even fewer 8\% of these registered deaths having a Cause of Death (COD) recorded \cite{worldhealthorganizationSCOREHealthData2021}. To fill this gap in death registrations, an alternative method known as Verbal Autopsy (VA) is used to collect data on deaths and determine their likely causes at scale \cite{desavignyIntegratingCommunitybasedVerbal2017,thomasVerbalAutopsyHealth2018,rampatigeSystematicReviewStatistics2014}, outside of traditional healthcare facilities where over half of deaths occur at home \cite{adairWhoDiesWhere2021}.

VA involves two major components: survey and COD assignment \cite{worldhealthorganizationVerbalAutopsyStandards2023,chandramohanEstimatingCausesDeath2021,worldhealthorganizationVerbalAutopsyStandards2007}. In the survey component, trained lay surveyors interview those familiar with the deceased (e.g. living spouse, children, family, friends) to gather information using standardized questionnaires and open narratives. In the COD assignment component, physicians evaluate information available from the questionnaires and open narratives to assign probable CODs. Although VA surveys have been an effective alternative to collect mortality data at scale (e.g. less than \$3 USD per house in India \cite{gomesNationwideMortalityStudies2017,jhaProspectiveStudyOne2005}), COD assignment has been criticized to be expensive and difficult to reproduce due to reliance on physician assignment \cite{mccormickProbabilisticCauseofDeathAssignment2016b,morrisFactorsAssociatedPhysician2010,solemanVerbalAutopsyCurrent2006}. As an alternative to physician assignment, computer models, such as InterVA \cite{byassIntegratedApproachProcessing2019} and InSilicoVA \cite{mccormickProbabilisticCauseofDeathAssignment2016b}, have recently been studied to automatically assign CODs with performances close to physicians at the population level, but poor performances at the individual level \cite{jhaAutomatedPhysicianAssignment2019,leitaoComparisonPhysiciancertifiedVerbal2014,desaiPerformanceFourComputercoded2014,tungaVerbalAutopsyModels2021,otiVerbalAutopsyInterpretation2010}. These computer models often utilized data from the structured questionnaire, but often omit the free-text open narrative, which misses latent information, such as chronology or health-seeking behaviors, that may potentially help models perform better than using the questionnaire alone \cite{jebleeAutomaticallyDeterminingCause2019,blancoExtractingCauseDeath2021,kingQualityDiagnosticValue2016}.

Recently, Large Language Models (LLM), leveraging massive datasets and deep learning approaches, have made advances in performing a variety of Natural Language Processing (NLP) tasks using free-text, such as question answering, code generation, and even medical diagnosis \cite{changSurveyEvaluationLarge2023,lundChattingChatGPTHow2023,svyatkovskiyIntelliCodeComposeCode2020,hauptAIGeneratedMedicalAdvice2023}. On November 30, 2022, a widely-available LLM called ChatGPT was released by OpenAI with capabilities of answering natural language text inquiries using training data up to September 2021. ChatGPT-3 was based on several Generative Pre-trained Transformer (GPT) models between 2018 to 2020, namely GPT-1 to GPT-3, which had notable differences in training data sizes of 5 gigabytes to 45 terabytes from web sources that resulted in 117 million to 175 billion parameter models \cite{wuBriefOverviewChatGPT2023}. On March 14, 2023, ChatGPT-4 was released with human-level performance on various professional and academic exams and benchmarks that outperformed ChatGPT-3 \cite{openaiGPT4TechnicalReport2023}. Given the limited usage of free-text open narratives in computer models for determining CODs, and recent advances in LLMs that leverage natural language text prompts, we conduct a case study with Sierra Leone deaths from VA in 2019 to 2022 to compare four models, GPT-3.5, GPT-4, InterVA-5, and InSilicoVA, to physicians for determining CODs.

\section{Methods}\label{methods}

This study uses 6939 physician agreed records of 11,920 records collected in 2019 to 2022 from the Healthy Sierra Leone (HEAL-SL) study as described in Section \ref{methods-data}. Section \ref{methods-models} describes the four models, GPT-3.5, GPT-4, InterVA-5, and InSilicoVA, for COD assignment, and their inputs and outputs, while section \ref{methods-eval} details the performance evaluation of the four models relative to physicians overall, by ages, and by COD, using population and individual level metrics. Figure \ref{fig-methods} details the methods used in this study.

\begin{figure}[h!]
\centering
\includegraphics[width=1\textwidth]{fig-data-sampling.pdf}
\caption{Case study sample areas. VA records were collected from sample areas in Sierra Leone between 2019-2022 from the HEAL-SL study. Sample deaths shown represent records with physician assigned CODs only.}\label{fig-data-sampling}
\end{figure}

\subsection{Verbal Autopsy (VA) Data}\label{methods-data}

Initially, 11,920 records\footnote{Available at openmortality.org/dataset/heal-sl} from the HEAL-SL study \cite{njalauniversityHealthySierraLeone2023,carshon-marshChildMaternalAdult2022} were collected from dual-coded EVA, where each record was randomly coded by two different physicians that assigned CODs as International Classification of Diseases Revision 10 (ICD-10) codes \cite{worldhealthorganizationICD10InternationalStatistical2011}. Physicians were able to assign CODs for 11,820 of the 11,920 records, where 100 of these records could not be assigned a COD due to missing or inadequate information (e.g. low quality narrative, data loss). To determine if two codes were in physician agreement per record, codes were compared for similarity using Central Medical Evaluation Agreement 10 (CMEA-10) codes, which groups a range of similar ICD-10 codes together indicating codes in agreement \cite{aleksandrowiczPerformanceCriteriaVerbal2014} (see Additional File 2). When codes were not in agreement, a record enters the reconciliation phase, where the two physicians were provided reasoning and initial codes from each other to: (1) keep their initial code (2) assign the other physician's code or (3) assign a new code. If codes were not in agreement after the reconciliation phase, a record enters the adjudication phase, where a third senior physician evaluates both physicians' reasoning and codes before and after reconciliation, and assigns a final code based on their evaluation. The 11,820 physician coded records were further filtered for records where both physicians agreed on the assigned codes (records that were not reconciled or adjudicated) resulting in 6942 physician agreed records. Since computer models were compared to physicians in this study, there was more certainty that COD assignments agreed by both physicians were representative of physician assignment than when they disagreed \cite{barnettComparativeAccuracyDiagnosis2019,morrisFactorsAssociatedPhysician2010,hsiaoFactorsAssociatedPhysician2012}. The 6942 records were converted into CGHR-10 codes (see Additional File 1) that generalized ICD-10 codes into 19, 10, and 7 categories for the adult (12 to 69 years), child (28 days to 11 years), and neonatal (under 28 days) age groups.  After conversion, a final total of 6939 physician agreed records (3826 adult, 2636 child, and 477 neonatal) were used for modelling and performance evaluation, where three records were removed as their ICD-10 codes did not have a matching CGHR-10 code.

The 6939 physician agreed records were relatively evenly distributed (approximately 40-57\% male and female) for the adult, child, and neonatal age groups in terms of sex (Figure \ref{fig-data-agesex}) with the most records in the 1-5 year (n=1633, 24\%) age range, and the least records in the 10-14 year (n=135, 2\%) and 7-27 day (n=82, 1\%) age ranges (Figure \ref{fig-data-agerange}). The other 15 five-year age ranges within 0 days to 69 years were relatively evenly distributed with approximately 4-6\% of the 6939 records each. The 3826 adult records had most records in the 65-69 year (n=575, 15\%), and the least in the 10-14 year (n=135, 4\%) age ranges, while the other 10 age ranges were relatively evenly distributed at 7-9\% of all records (Figure \ref{fig-data-agerange-adult}). For adult CODs, Malaria (n=799, 21\%) had the highest number of adult records, and cancers (n=49, 1\%), diabetes mellitus (n=27, \textless 1\%), and suicide (n=3, \textless 1\%) had the lowest number adult of adult records, while the other 15 CODs had between 2-11\% of all adult records (Figure \ref{fig-data-cod-adult}). The 2636 child records had most records in the 1-5 year (n=1633, 62\%) age range while the three other age ranges were relatively evenly distributed at 12-14\% of all records (Figure \ref{fig-data-agerange-child}). For child CODs, Malaria (n=1382, 52\%) and other infections (n=667, 25\%) had the highest number of child records, while nutritional deficiencies (n=11, \textless 1\%) and congenital anomalies (n=1, \textless 1\%) had the lowest number of records with the five other CODs between 3-7\% of all child records. The 477 neonatal records had two age ranges with most records in the 0-6 day (n=395, 83\%) age range and the rest in the 7-27 day (n=82, 17\%) age range (Figure \ref{fig-data-agerange-neo}). For neonatal CODs, stillbirth (n=172, 36\%) had the highest number of neonatal records, and ill-defined (n=23, 5\%), other (n=5, 1\%), and congenital anomalies (n=2, \textless 1\%) had the least number of records with the rest of the other three CODs between 15-22\% of all neonatal records (Figure \ref{fig-data-cod-neo}). See Appendices \ref{app-data} to \ref{app-data-neo} for a list of figures \ref{fig-data-agesex} to \ref{fig-data-cod-neo} showing distributions of the 6939 physician agreed records.

\subsection{Modelling}\label{methods-models}

Four computer models were used to assign COD for each of the 6939 physician agreed records: GPT-3.5, GPT-4, InterVA-5, and InSilicoVA. Each model required pre-processing of the 6939 records into input data, and standardization of output COD codes from models for performance evaluation as not all models produced comparable codes across outputs. Although each model can assign multiple CODs per record, only the first generated COD response from GPT-3.5 and GPT-4, and the most probable COD from InterVA-5 and InSilicoVA were used for evaluation. All model outputs were converted to CGHR-10 codes to evaluate performances of models for COD assignment relative to physicians.

\subsubsection{Overview of GPT-3.5/4, InterVA-5, and InSilicoVA}\label{methods-models-overview}

GPT-3.5 \cite{brownLanguageModelsAre2020} and GPT-4 \cite{openaiGPT4TechnicalReport2023} are LLMs that utilize deep neural networks with transformer architectures \cite{vaswaniAttentionAllYou2017} and reinforcement learning from human feedback \cite{ouyangTrainingLanguageModels2022,christianoDeepReinforcementLearning2017,stiennonLearningSummarizeHuman2020,wirthSurveyPreferencebasedReinforcement2017} to follow instructions from prompts and provide human-level responses, with known differences in GPT-4 possessing multimodal capabilites (e.g. image/voice input/output), more recent training data, and improved responses compared to ChatGPT-3 \cite{wuBriefOverviewChatGPT2023}. GPT-3.5 and GPT-4 were instructed with text prompts to assign CODs based on the open narrative. InterVA-5 and InSilicoVA are widely used and studied standard statistical models \cite{murrayUsingVerbalAutopsy2014,otiVerbalAutopsyInterpretation2010,benaraEvaluationMethodsAssigning2023,chandramohanEstimatingCausesDeath2021,tungaVerbalAutopsyModels2021,leitaoComparisonPhysiciancertifiedVerbal2014,jhaAutomatedPhysicianAssignment2019} for COD assignment in VAs under the openVA framework \cite{liOpenVAToolkitVerbal2023}. InterVA-5 applies Bayesian probabilistic modelling \cite{bayesEssaySolvingProblem1958} using a set of standardized symptoms from reports and related conditional probabilities from medical experts to assign CODs based on the highest probability \cite{byassIntegratedApproachProcessing2019,byassStrengtheningStandardisedInterpretation2012}. InSilicoVA improves upon InterVA (e.g. comparable probabilities across individuals, measures of uncertainty, and inclusion of additional data sources) with a hierarchical Bayesian framework and Markov Chain Monte Carlo (MCMC) simulations \cite{brooksMarkovChainMonte1998,chibMarkovChainMonte2001,hanMarkovChainMonte2001} to incorporate multiple sources of uncertainty for assigning CODs based on the highest probability \cite{mccormickProbabilisticCauseofDeathAssignment2016b}. For assigning CODs, GPT-3.5 and GPT-4 require prompts containing conversation-like textual instructions as input, while InterVA-5 and InSilicoVA require structured symptom and associated sociodemographic data as input. 

\subsubsection{Input Data and Preprocessing}\label{methods-models-input}

For GPT-3.5 and GPT-4, 6939 text prompts were generated for each physician agreed record as input to instruct the models to assign CODs based on the open narratives. Two types of text prompts were used: user prompts and system prompts. System prompts contained textual instructions to assign the role of a physician ICD-10 coder with expertise in Sierra Leone. The following system prompt was used for each record:

\begin{lstlisting}
You are a physician with expertisein determining underlying causes of death in Sierra Leone by assigning the most probable ICD-10 code for each death using verbal autopsy narratives. Return only the ICD-10 code without description. E.g. A00. If there are multiple ICD-10 codes, show one code per line.
\end{lstlisting}

\noindent User prompts contained textual instructions to perform coding of VA records based on the age, sex, and narrative of the deceased. The following template was used to generate user prompts for each record, where \verb|<age>| and \verb|<sex>| from the questionnaire, and \verb|<narrative>| from the narratives, were replaced with values from the data:

\begin{lstlisting}
Determine the underlying cause of death and provide the most probable ICD-10 code for a verbal autopsy narrative of a <age> years old <sex> death in Sierra Leone:  <narrative>
\end{lstlisting}

\noindent For InterVA-5 and InSilicoVA, the standardized questionnaire data from the HEAL-SL EVA were first converted into 2016 World Health Organization (WHO) VA questionnaire revision 1.5.1 Open Data Kit (ODK) format \cite{worldhealthorganizationODKVerbalAutopsy2022,nafundiODKCollectData2023} consisting of 526 variables \cite{dipasqualeReleaseODK20162016}, followed by further conversion into OpenVA format \cite{liOpenVAToolkitVerbal2023} consisting of 353 variables \cite{byassInterVA5UserGuide2020} using the \verb|pyCrossVA| version 0.97 Python package \cite{thomasPycrossvaPrepareData}. The 6939 records were all converted into OpenVA formatted records for InterVA-5 and InSilicoVA.

\subsubsection{Models and Parameters}\label{methods-models-params}

The GPT-3.5 and GPT-4 Application Programming Interface (API) was accessed using Python version 3.11.4 and used to assign CODs for each record. GPT-3.5 used the \verb|gpt-3.5-turbo| model, while GPT-4 used the \verb|gpt-4-0613| model. The parameter \verb|temperature| for GPT-3.5 and GPT-4, representing the sampling temperature ranging from 0 to 2 (default of 1), was set to 0 to produce more deterministic outputs \cite{openaiOpenAIPlatformAPI2024}. Higher values closer to 2 may produce less deterministic outputs, while lower values closer to 0 produce more deterministic outputs.

The \verb|openVA| R package was used to run InterVA-5 and InSilicoVA models to assign CODs for each record in R version 4.3.1. The \verb|openVA| package version 1.1.1 used dependent packages \verb|InterVA5| version 1.1.3 and \verb|InSilicoVA| version 1.4.0. The \verb|Nsim| (number of iterations to run) parameter \cite{liInSilicoVAProbabilisticVerbal2022} for InSilicoVA was set to 9500\footnote{The default value of Nsim=10000 for InSilicoVA ran until 9500 iterations before it stopped due to errors, thus Nsim=9500 was used and ran successfully for all iterations.}, while the \verb|HIV| (level of prevalence of human immunodeficiency virus) and \verb|Malaria| (level of prevalence of Malaria) parameters \cite{thomasInterVA5ReplicateAnalyse2021} for InterVA-5 were both set to \verb|'h'| (high) reflecting HIV and Malaria disease assumptions in Sierra Leone \cite{yendewaHIVAIDSSierra2018,walkerMalariaMorbidityMortality2015}.

\subsubsection{Output Data and Code Conversions}\label{methods-models-output}

Of the 6939 input records, GPT-3.5, GPT-4, InterVA-5, and InSilicoVA were able to assign CODs for 6939 (100\%), 6935 (\textgreater 99\%), 6830 (98\%), 6830 (98\%) records respectively. All 6830 (100\%) InterVA-5 and InSilicoVA records with WHO VA 2016 v1.5 output codes \cite{worldhealthorganizationVerbalAutopsyStandards2016} were converted into ICD-10 codes respectively. After all model outputs were converted to ICD-10 codes, they were further converted to CGHR-10 codes. The 6939 GPT-3.5 and 6935 GPT-4 output records with ICD-10 codes were converted into 6930 (\textgreater 99\%) and 6931 (\textgreater 99) records with CGHR-10 codes, where \textless 1\% (9 and 8) records did not have matching CGHR-10 codes respectively. The 6830 InterVA-5 and InSilicoVA records with ICD-10 codes were converted into 6802 (\textgreater 99\%) and 6726 (98\%) records with CGHR-10 codes respectively, where 28 (\textless 1\%) and 104 (1\%) of records could not be converted into CGHR-10 codes.

\subsection{Performance Evaluation}\label{methods-eval}

The performance of four models, GPT-3.5, GPT-4, InterVA-5, and InSilicoVA, were evaluated with metrics on the population and individual level by comparing their CGHR-10 COD outputs to 6939 records. Cause Specific Mortality Fraction (CSMF) accuracy was used to evaluate models on the population level, while Partial Chance Corrected Concordance (PCCC) was used to evaluate models on the individual level \cite{murrayRobustMetricsAssessing2011}. Records that were assigned a COD by physicians, but not by a model were considered to be an incorrect COD assignment by the model. CSMF accuracy and PCCC were calculated for each model by three age groups, and further into age ranges and COD for each age group.

\subsubsection{Cause Specific Mortality Fraction (CSMF) Accuracy}\label{methods-eval-csmfa}

CSMF accuracy measures the performance of models at the population level, comparing distributions of CODs between the physicians and the models \cite{murrayRobustMetricsAssessing2011}. To calculate CSMF accuracy, we first calculate $CSMF_j$ as is the fraction of physician or model records for cause $j$, given by dividing the number of records for cause $j$ with the total number of records as seen in Equation \ref{eq-csmf}. Then, the $CSMFMaximumError$, representing the worst possible model, is calculated using Equation \ref{eq-csmferr}. Finally, the CSMF accuracy is given by Equation \ref{eq-csmfa}, where $k$ is the number of causes, $j$ is a cause, $CSMF^{true}_j$ is the true physician CSMF for cause $j$, and $CSMF^{pred}_j$ is the prediction model CSMF for cause $j$. CSMF accuracy ranges from 0 to 1, where 1 means that the model completely matched the physician COD distribution and 0 means that it did not match the distribution at all.

\begin{equation}\label{eq-csmf}
    CSMF_j = Records_j / Records
\end{equation}

\begin{equation}\label{eq-csmferr}
    CSMFMaximumError = 2(1 - Min(CSMF^{true}_j)
\end{equation}

\begin{equation}\label{eq-csmfa}
    CSMFAccuracy = 1 - \frac{\sum^{k}_{j=1}|CSMF^{true}_j - CSMF^{pred}_j|}{CSMFMaximumError}
\end{equation}

\subsubsection{Partial Chance Corrected Concordance (PCCC)}\label{methods-eval-pccc}

PCCC measures the performance of models at the individual level, comparing COD assignments between the physicians and models on a record by record basis, correcting for COD assignments made purely by chance \cite{murrayRobustMetricsAssessing2011}. PCCC is given by Equation \ref{eq-pccc}, where $k$ is the number of top COD assignments from the model to consider, $N$ is number of causes, and $C$ is fraction of records where the physician COD assignment is one of the top COD assignments from the model. For this study, we set $k$ to 1, making $C$ equivalent to the fraction of true positives $TP$ or records where the physician COD assignment is equal to the model COD assignment as shown in Equation \ref{eq-pccc-c}. Higher PCCC values closer to 1 indicate that model COD assignments are similar to physician COD assignments, while values closer to 0 indicate that model COD assignments are not similar to physicians.

\begin{equation}\label{eq-pccc-c}
    C = \frac{TP}{Records}
\end{equation}

\begin{equation}\label{eq-pccc}
    PCCC(k) = \frac{C - \frac{k}{N}}{1 - \frac{k}{N}}
\end{equation}

\subsubsection{Metrics Overall and by Age and Cause of Death}\label{methods-eval-agecod}

The CSMF accuracy and PCCC metrics were calculated and compared for each model overall and by age group, followed by age ranges and COD for each age group as model performance can vary across ages and specific causes \cite{benaraEvaluationMethodsAssigning2023,murrayUsingVerbalAutopsy2014,setelValidityVerbalAutopsy2006}. Metrics were calculated overall for three age groups according to the CGHR-10 codes: adult (12 to 69 years), child (28 days to 11 years), and neonatal (under 28 days). For each of the adult and child age groups, metrics were calculated for five-year age ranges for records with ages at death of one-year or older and five-month age ranges for 28 days or older. For the neonatal age group, the age ranges of 0-6 days and 7-27 days were used. Metrics were also calculated by CODs defined by CGHR-10 codes, which include 19, 10, and 7 CODs for adult, child, and neonatal records respectively.

\section{Results}\label{results}

This section details the performance results of GPT-3.5, GPT-4, InterVA-5, and InSilicoVA models for assigning CGHR-10 CODs after applying the methods in Section \ref{methods}. GPT-4 performed the best overall at 0.71 PCCC followed by GPT-3.5 at 0.64 PCCC. GPT-4 also had the highest PCCC for most age ranges and CODs across the adult (12 to 69 years), child (28 days to 11 years), and neonatal (under 28 days) age groups with GPT-3.5, InterVA-5, and InSilicoVA having higher PCCC values for some age ranges and CODs. Overall performance results are seen in Section \ref{results-overall}, and performance by adult, child, and neonatal records are seen in Sections \ref{results-adult}, \ref{results-child}, and \ref{results-neo} respectively.

\subsection{Overall Performance}\label{results-overall}

Of the 6939 records, GPT-4 had the highest PCCC followed by GPT-3.5, InSilicoVA, and InterVA-5 at 0.61, 0.58, 0.44, and 0.43 PCCC respectively (Table \ref{tab-perf-overall}). GPT-3.5 and GPT-4 had improvements ranging from 0.14-0.18 PCCC over InSilicoVA and InterVA-5, while GPT-4 slightly improved over GPT-3.5 by 0.03 PCCC. CSMF accuracies were above 0.7 for all models at 0.79, 0.74, 0.74, and 0.78 for GPT-4, GPT-3.5, InSilicoVA, and InterVA-5 respectively. Figure \ref{fig-perf-agegroup} shows the PCCC performance across age groups with reference to the CSMF Accuracy. GPT-4 had the highest PCCC at 0.65 and 0.62 for the adult and neonatal age groups, while GPT-3.5 had the highest PCCC at 0.57 for the child age group with GPT-4 performing slightly worse at 0.54. GPT-3.5 had the second highest PCCC at 0.56 fo the adult age group, while InSilicoVA had the second highest PCCC for the neonatal age group at 0.56. InSilicoVA and InterVA-5 performed the worse for both adult and child age groups at PCCC scores below 0.5, while GPT-3.5 performed the worse for the neonatal age group with a PCCC of 0.42. The PCCC for models had larger differences between the minimum and maximum PCCC for the adult age group at 0.24, than the child and neonatal age groups at 0.12 and 0.20 respectively. For reference, all records, including non-physician agreed records, were also coded by GPT-3.5, GPT-4, InterVA-5, and InSilicoVA for comparison to physician agreed records. For physician agreed records, models had 0.05 to 0.09 higher PCCC values and 0.01 to 0.05 higher CSMF accuracy values than for all records (see Figure \ref{fig-perf-allvsagree} in Appendix \ref{app-perf-misc}).

\begin{table}[h]
\caption{Overall model performance for physician agreed records (n=6939)}\label{tab-perf-overall}%
\begin{tabular}{@{}llll@{}}
\toprule
Model & PCCC & CSMF Accuracy & PCCC Improvement\\
\midrule
GPT-4 & 0.61 & 0.79 & $\uparrow$ 0.03\\
GPT-3.5 & 0.58 & 0.74 & $\uparrow$ 0.14\\
InSilicoVA & 0.44 & 0.74 & $\uparrow$ 0.01\\
InterVA-5 & 0.43 & 0.78 & -\\
\botrule
\end{tabular}
\footnotetext{Note: Models sorted from highest to lowest PCCC. PCCC improvements are from model below.}
\end{table}

\begin{figure}[h!]
\centering
\includegraphics[width=0.7\textwidth]{fig-perf-agegroup.pdf}
\caption{Model performance by age group. GPT-3.5, GPT-4, InterVA-5, and InSilicoVA model performance for assigning CGHR-10 CODs in the adult (12 to 69 Years), child (28 days to 11 years), and neonatal (under 28 days) age groups. CSMF indicates CSMF accuracy.}\label{fig-perf-agegroup}
\end{figure}

\clearpage
\subsection{Performance for 3826 Adult Records (12 to 69 years)}\label{results-adult}

Across the adult age ranges, GPT-4 had the highest PCCC values at or above 0.59, followed by GPT-3.5, while InterVA-5 and InSilicoVA had the lowest PCCC values at or below 0.46. GPT-4 PCCC was between 0.61 and 0.72 for young to middle age ranges between 10 to 49 years, and at or below 0.62 for older age ranges between 50 to 69 years (Figure \ref{fig-perf-agerange-adult}). Each age range had between 135 (4\%) to 575 (15\%) records with the 65-69 year age range having the most records (n=575, 15\%), while other age ranges ranged from 4-9\% of all records. Figure \ref{fig-perf-cod-adult} shows the highest and lowest performing models by PCCC across 18 adult CODs. The highest PCCC values were GPT-4 between 0.36 to 0.99 for 12 of 18 CODs (GPT-3.5 within 0.05 for 7 of the 15 CODs), InSilicoVA for road and transport injuries at 0.84 (InterVA-5 close at 0.81), GPT-3.5 for ill-defined at 0.57,and InterVA-5 for ischemic heart disease at 0.61 (GPT-4 close at 0.57). The lowest PCCC values were InterVA-5 between 0 to 0.79 for 11 of 18 CODs (InSilicoVA within 0.05 for 3 of the 11 CODs), InSilicoVA between 0.01 to 0.41 for 6 of 18 CODs (InterVA-5 within 0.05 for other cardiovascular diseases and diarrhoeal diseases), and GPT-3.5 for ischemic heart disease at 0.49 (InSilicoVA within 0.05) and road and transport injuries at 0.59. Suicide and chronic respiratory diseases had the largest differences between minimum and maximum PCCC at 1 and 0.81 respectively. Maternal conditions, road and transport injuries, and tuberculosis had higher PCCC values above or at 0.67 (with the exception of GPT-3.5 for road and transportation injuries as an outlier), where there were small differences between minimum and maximum PCCC ranging from 0.12 to 0.19. Ischemic heart disease and unspecified infections had lower PCCC values below or at 0.6, where there were small differences between minimum and maximum PCCC ranging at 0.02 and 0.17 respectively. Each adult COD had between 3 (\textless 1\%) to 799 records (21\%) with malaria (n=799, 21\%), other noncommunicable diseases (n=414, 11\%), and unspecified infections (n=389, 10\%) making 42\% of all records, suicide having less than 1\% of records, and other CODs between 1-9\% of all records. GPT-4 had the highest PCCC for both male and female records at 0.72 and 0.7 respectively, while all other models had PCCC at or below 0.63 (see Figure \ref{fig-perf-sex-adult} in Appendix \ref{app-perf-misc}).

\begin{figure}[h!]
\centering
\includegraphics[width=0.7\textwidth]{fig-perf-agerange-adult.pdf}
\caption{Model performance for adult records by age range. GPT-3.5, GPT-4, InterVA-5, and InSilicoVA model performance for assigning CGHR-10 CODs in the adult age group (12 to 69 Years) by five-year age ranges. PCCC values are sorted by the highest to lowest PCCC for each age range. Only models with PCCC values within .05 of the highest and lowest PCCC are shown.}\label{fig-perf-agerange-adult}
\end{figure}

\begin{figure}[h!]
\centering
\includegraphics[width=0.7\textwidth]{fig-perf-cod-adult.pdf}
\caption{Model performance for adult records by COD. GPT-3.5, GPT-4, InterVA-5, and InSilicoVA model performance for assigning CGHR-10 CODs in the adult age group (12 to 69 Years) by COD. PCCC values are sorted by the highest to lowest PCCC for each COD. Only models with PCCC values within .05 of the highest and lowest PCCC are shown.}\label{fig-perf-cod-adult}
\end{figure}

\clearpage
\subsection{Performance for 2636 Child Records (28 Days to 11 Years)}\label{results-child}

Across the child age ranges, GPT had the highest PCCC at 0.69 to 0.75 for all age ranges (GPT-3.5 within 0.05 for 1-11 years), while the lowest PCCC values were for InSilicoVA for 1-11 years (InterVA-5 within 0.05 for 1-5 years), and InterVA-5 for 1-11 months (Figure \ref{fig-perf-agerange-child}). Each child age range had between 308 (12\%) to 1633 (62\%) records with most records in the 1-5 years range at 1633 records (62\%) and the other age ranges consisting of 12-14\% of all records. Figure \ref{fig-perf-cod-child} shows the highest and lowest performing models by PCCC across 9 child CODs. GPT-4 had the highest PCCC for 7 of 9 child CODs, where GPT-3.5 had 5, InterVA-5 had two (congenital anomalies and diarrhoeal diseases), and InSilicoVA had one (pneumonia) of the 7 CODs within 0.05 PCCC of GPT-4. InterVA-5 had the highest PCCC for diarrhoeal disease at 0.82 (GPT-4 at 0.81), and GPT-3.5 had the highest PCCC for ill-defined at 0.84. InSilicoVA had the lowest PCCC for 4 of 9 child CODs (InterVA-5 within 0.05 for 3 of the 4 CODs), InterVA-5 had the lowest PCC for the other 4 of 9 child CODs (InSilicoVA within 0.05 for injuries), and GPT-3.5 had the lowest PCCC for diarrhoeal diseases at 0.69. Congenital anomalies had one record only in which GPT-4, GPT-3.5, and InterVA-5 assigned a CGHR-10 code matching the physicians. Ill-defined had the largest difference between minimum and maximum PCCC at 0.84, while injuries, diarrhoeal disease, and pneumonia (with the exception of InterVA-5 as an outlier at 0.41) had higher PCCC values at or above 0.69 with small differences between minimum and maximum PCCC ranging from 0.01 to 0.14. Other infections had lower PCCC values at or below 0.54 with a small difference between minimum and maximum PCCC at 0.2. Each child COD had between 1 (\textless 1\%) to 1382 (52\%) records with malaria (n=1382, 52\%), other infections (n=667, 25\%), and diarrhoeal diseases (n=175, 7\%) making 84\% of all records. Congenital anomalies and nutritional deficiencies had less than 1\% of all records, while other CODs had between 1-3\% of all records. GPT-4 had the highest PCCC for both male and female records at 0.73 and 0.72 (GPT-3.5 at 0.69) respectively, while all other models had PCCC at or below 0.68 (see Figure \ref{fig-perf-sex-child} in Appendix \ref{app-perf-misc}).

\begin{figure}[h!]
\centering
\includegraphics[width=0.7\textwidth]{fig-perf-agerange-child.pdf}
\caption{Model performance for child records by age range. GPT-3.5, GPT-4, InterVA-5, and InSilicoVA model performance for assigning CGHR-10 CODs in the child age group (28 days to 11 years) by five-year age ranges. PCCC values are sorted by the highest to lowest PCCC for each age range. Only models with PCCC values within .05 of the highest and lowest PCCC are shown.}\label{fig-perf-agerange-child}
\end{figure}

\begin{figure}[h!]
\centering
\includegraphics[width=0.7\textwidth]{fig-perf-cod-child.pdf}
\caption{Model performance for child records by COD. GPT-3.5, GPT-4, InterVA-5, and InSilicoVA model performance for assigning CGHR-10 CODs in the child age group (28 days to 11 years) by COD. PCCC values are sorted by the highest to lowest PCCC for each COD. Only models with PCCC values within .05 of the highest and lowest PCCC are shown.}\label{fig-perf-cod-child}
\end{figure}

\clearpage
\subsection{Performance for 477 Neonatal Records (Under 28 Days)}\label{results-neo}

GPT-4 and InSilicoVA had the highest PCCC for 0-6 day and 7-27 day neonatal age ranges at 0.62 and 0.67 respectively, while InterVA-5 had the lowest PCCC across all neonatal age ranges (GPT-3.5 within 0.05) as seen in Figure \ref{fig-perf-agerange-neo}. There were small differences between minimum and maximum PCCC across all age ranges at 0.14 and 0.12 for 0-6 days and 7-27 days respectively. Most records were in the 0-6 day age range at 395 (83\%) records, while the rest were in the 7-27 day age range at 82 (17\%) records. Figure \ref{fig-perf-cod-neo} shows the performance of models by PCCC across 7 neonatal CODs. GPT-4 had the highest PCCC for 4 of the 7 CODs (GPT-3.5 identical to GPT-4 for congenital anomalies at 1 PCCC). InSilicoVA, GPT-3.5, and InterVA-5 had the highest PCCC for neonatal infections at 0.87, ill-defined at 0.59 (GPT-4 at 0.55), and birth asphyxia and birth trauma at 0.59 respectively. Congenital anomalies and other had less or equal to 5 records with large difference of 1 between minimum and maximum PCCC. Neonatal infections and stillbirth had smaller differences between minimum and maximum PCCC at 0.38 and 0.24 respectively with larger PCCC values at or above 0.49. Birth asphyxia and birth trauma also had smaller differences between minimum and maximum PCCC at 0.24 but had PCCC values at or below 0.59. Prematurity and low birthweight and ill-defined had large differences between minimum and maximum PCCC at 0.62 and 0.58 respectively. Each neonatal COD had between 2 (\textless 1\%) to 172 (38\%) records, with stillbirth (n=172, 36\%), birth asphyxia and birth trauma (n=103, 22\%), and neonatal infections (n=99, 21\%), and prematurity and low birthweight (n=73, 15\%) making 94\% of all records, while the rest are in congenital anomalies, other, and ill-defined CODs. GPT-4 had the highest PCCC for both female and male records at 0.65 and 0.58 (InSilicoVA at 0.56) respectively, while all other models had PCCC at or below 0.56 (see Figure \ref{fig-perf-sex-neo} in Appendix \ref{app-perf-misc}).

\begin{figure}[h!]
\centering
\includegraphics[width=0.7\textwidth]{fig-perf-agerange-neo.pdf}
\caption{Model performance for neonatal records by age range. GPT-3.5, GPT-4, InterVA-5, and InSilicoVA model performance for assigning CGHR-10 CODs in the neonatal age group (under 28 days) by five-year age ranges. PCCC values are sorted by the highest to lowest PCCC for each age range. Only models with PCCC values within .05 of the highest and lowest PCCC are shown.}\label{fig-perf-agerange-neo}
\end{figure}

\begin{figure}[h!]
\centering
\includegraphics[width=0.7\textwidth]{fig-perf-cod-neo.pdf}
\caption{Model performance for neonatal records by COD. GPT-3.5, GPT-4, InterVA-5, and InSilicoVA model performance for assigning CGHR-10 CODs in the neonatal age group (under 28 years) by COD. PCCC values are sorted by the highest to lowest PCCC for each COD. Only models with PCCC values within .05 of the highest and lowest PCCC are shown.}\label{fig-perf-cod-neo}
\end{figure}

\clearpage
\section{Discussion}\label{discuss}

TO DO.

\subsection{Advantages}\label{discuss-adv}

x.

Captured causes of death with very low number of records.

\subsection{Disadvantages}\label{discuss-disadv}

x.

Dimensions, reliabiity, and cost of LLM versus physician.

Reliance on third-party non-open source models.

\subsection{Limitations}\label{discuss-limits}

x.

Limited records for some causes of death. Did not explore adjudicated and reconciliated records.

Did not explore effects of model parameters outside the default (e.g. ChatGPT temperature, etc).

Did not explore response consistency with reruns.

\subsection{Opportunities}\label{discuss-opp}

x.

Prompt engineering: adjust and optimize prompts, add in correctional instructions, adjudication, reconciliation, ICD-10/physician coding guide standards, and system prompt.

Sensitivity analysis: Test consistency of GPT responses with multiple reruns and parameter settings.

\section{Conclusion}\label{conclude}

TO DO.

% Conclusions may be used to restate your hypothesis or research question, restate your major findings, explain the relevance and the added value of your work, highlight any limitations of your study, describe future directions for research and recommendations. 

% In some disciplines use of Discussion or 'Conclusion' is interchangeable. It is not mandatory to use both. Please refer to Journal-level guidance for any specific requirements. 

\backmatter

\bmhead{Supplementary information}

% If your article has accompanying supplementary file/s please state so here. 

% Authors reporting data from electrophoretic gels and blots should supply the full unprocessed scans for key as part of their Supplementary information. This may be requested by the editorial team/s if it is missing.

% Please refer to Journal-level guidance for any specific requirements.

Additional files were used to supplement this paper:

\begin{itemize}
    \item Additional file 1: Centre for Global Health Research 10 (CGHR-10) codes. Codes grouping ICD-10 code ranges into generalized categories. (.csv)
    \item Additional file 2: Central Medical Evaluation Agreement 10 (CMEA-10) codes. ICD-10 code ranges considered in physician agreement. (.csv)
\end{itemize}

\bmhead{Acknowledgments}

% Acknowledgments are not compulsory. Where included they should be brief. Grant or contribution numbers may be acknowledged.

% Please refer to Journal-level guidance for any specific requirements.

\section*{Declarations}

% Some journals require declarations to be submitted in a standardised format. Please check the Instructions for Authors of the journal to which you are submitting to see if you need to complete this section. If yes, your manuscript must contain the following sections under the heading `Declarations':

% \begin{itemize}
% \item Funding
% \item Conflict of interest/Competing interests (check journal-specific guidelines for which heading to use)
% \item Ethics approval 
% \item Consent to participate
% \item Consent for publication
% \item Availability of data and materials
% \item Code availability 
% \item Authors' contributions
% \end{itemize}

% \noindent
% If any of the sections are not relevant to your manuscript, please include the heading and write `Not applicable' for that section. 

\subsection*{Availability of data and materials}

The datasets supporting the conclusions of this article are included within the article (and its additional files).

\clearpage
\begin{appendices}

%\section{Section title of first appendix}\label{secA1}

%An appendix contains supplementary information that is not an essential part of the text itself but which may be helpful in providing a more comprehensive understanding of the research problem or it is information that is too cumbersome to be included in the body of the paper.
    
%%=============================================%%
%% For submissions to Nature Portfolio Journals %%
%% please use the heading ``Extended Data''.   %%
%%=============================================%%

%%=============================================================%%
%% Sample for another appendix section			       %%
%%=============================================================%%

%% \section{Example of another appendix section}\label{secA2}%
%% Appendices may be used for helpful, supporting or essential material that would otherwise 
%% clutter, break up or be distracting to the text. Appendices can consist of sections, figures, 
%% tables and equations etc.

\section{Case Study Methods}

\begin{figure}[h!]
\centering
\includegraphics[width=0.625\textwidth]{fig-methods.pdf}
\caption{Case study methods. VA records in Sierra Leone between 2019-2022 from the HEAL-SL study were filtered for physician agreed records, and pre-processed for COD assignment by GPT-3.5, GPT-4, InterVA-5 and InSilicoVA models. The output models were evaluated with CSMF Accuracy and PCCC overall, by age group, and by COD and five-year age ranges for each age group.}\label{fig-methods}
\end{figure}

\clearpage
\section{Physician Agreed Verbal Autopsy Records}\label{app-data}

The following figures visualize the distribution of all physician agreed records by age groups (Figure \ref{fig-data-agesex}) and 5-year age ranges (Figure \ref{fig-data-agerange}) in Sierra Leone from 2019 to 2022.

\begin{figure}[h!]
\centering
\includegraphics[width=1\textwidth]{fig-data-agesex.pdf}
\caption{Number of physician agreed VA records by adult, child, and neonatal age groups in Sierra Leone from 2019 to 2022.}\label{fig-data-agesex}
\end{figure}

\newpage
\begin{figure}[h!]
\centering
\includegraphics[width=0.85\textwidth]{fig-data-agerange.pdf}
\caption{Number of physician agreed VA records by 5-year age ranges in Sierra Leone from 2019 to 2022.}\label{fig-data-agerange}
\end{figure}


\clearpage
\section{Adult Physician Agreed Verbal Autopsy Records (12 to 69 Years)}\label{app-data-adult}

The following figures visualize the distribution of adult physician agreed records by 5-year age ranges (Figure \ref{fig-data-agerange-adult}) and CGHR-10 COD code in Sierra Leone from 2019 to 2022 (Figure \ref{fig-data-cod-adult}).

\begin{figure}[h!]
\centering
\includegraphics[width=0.95\textwidth]{fig-data-agerange-adult.pdf}
\caption{Number of physician agreed adult (12 to 69 years) VA records by 5-year age ranges in Sierra Leone from 2019 to 2022.}\label{fig-data-agerange-adult}
\end{figure}

\begin{figure}[h!]
\centering
\includegraphics[width=0.85\textwidth]{fig-data-cod-adult.pdf}
\caption{Number of physician agreed adult (12 to 69 years) VA records by CGHR-10 COD code in Sierra Leone from 2019 to 2022.}\label{fig-data-cod-adult}
\end{figure}

\clearpage
\section{Child Physician Agreed Verbal Autopsy Records (28 Days to 11 Years)}\label{app-data-child}

The following figures visualize the distribution of child physician agreed records by 5-year age ranges (Figure \ref{fig-data-agerange-child}) and CGHR-10 COD code in Sierra Leone from 2019 to 2022 (Figure \ref{fig-data-cod-child}).

\begin{figure}[h!]
\centering
\includegraphics[width=0.7\textwidth]{fig-data-agerange-child.pdf}
\caption{Number of physician agreed child (28 days to 11 years) VA records by 5-year age ranges in Sierra Leone from 2019 to 2022.}\label{fig-data-agerange-child}
\end{figure}

\begin{figure}[h!]
\centering
\includegraphics[width=0.85\textwidth]{fig-data-cod-child.pdf}
\caption{Number of physician agreed child (28 days to 11 years) VA records by CGHR-10 COD code in Sierra Leone from 2019 to 2022.}\label{fig-data-cod-child}
\end{figure}

\clearpage
\section{Neonatal Physician Agreed Verbal Autopsy Records (Under 28 Days)}\label{app-data-neo}

The following figures visualize the distribution of neonatal physician agreed records by 5-year age ranges (Figure \ref{fig-data-agerange-neo}) and CGHR-10 COD code in Sierra Leone from 2019 to 2022 (Figure \ref{fig-data-cod-neo}).

\begin{figure}[h!]
\centering
\includegraphics[width=0.75\textwidth]{fig-data-agerange-neo.pdf}
\caption{Number of physician agreed neonatal (less than 28 days) VA records by 5-year age ranges in Sierra Leone from 2019 to 2022.}\label{fig-data-agerange-neo}
\end{figure}

\begin{figure}[h!]
\centering
\includegraphics[width=0.85\textwidth]{fig-data-cod-neo.pdf}
\caption{Number of physician agreed neonatal (less than 28 days) VA records by CGHR-10 COD code in Sierra Leone from 2019 to 2022.}\label{fig-data-cod-neo}
\end{figure}

\clearpage
\section{Performance By Other Variables}\label{app-perf-misc}

The following figures show the model performance by PCCC for all records and physician agreed records (Figure \ref{fig-perf-allvsagree}), and the performance of models by PCCC across sex for the adult (12 to 69 years), child (28 days to 11 years), and neonatal (under 28 days) age groups in Figures \ref{fig-perf-sex-adult}, \ref{fig-perf-sex-child}, and \ref{fig-perf-sex-neo} respectively.

\begin{figure}[h!]
\centering
\includegraphics[width=0.85\textwidth]{fig-perf-allvsagree.pdf}
\caption{Model performance for all records versus physician agreed records. GPT-3.5, GPT-4, InterVA-5, and InSilicoVA model performance for assigning CGHR-10 CODs for all records and physician agreed records. PCCC values are sorted by the highest to lowest PCCC.}\label{fig-perf-allvsagree}
\end{figure}

\begin{figure}[h!]
\centering
\includegraphics[width=0.85\textwidth]{fig-perf-sex-adult.pdf}
\caption{Model performance for adult records by sex. GPT-3.5, GPT-4, InterVA-5, and InSilicoVA model performance for assigning CGHR-10 CODs in the adult age group (12 to 69 years) by sex. PCCC values are sorted by the highest to lowest PCCC.}\label{fig-perf-sex-adult}
\end{figure}

\begin{figure}[h!]
\centering
\includegraphics[width=0.85\textwidth]{fig-perf-sex-child.pdf}
\caption{Model performance for child records by sex. GPT-3.5, GPT-4, InterVA-5, and InSilicoVA model performance for assigning CGHR-10 CODs in the child age group (28 days to 11 years) by sex. PCCC values are sorted by the highest to lowest PCCC.}\label{fig-perf-sex-child}
\end{figure}

\begin{figure}[h!]
\centering
\includegraphics[width=0.85\textwidth]{fig-perf-sex-neo.pdf}
\caption{Model performance for neonatal records by sex. GPT-3.5, GPT-4, InterVA-5, and InSilicoVA model performance for assigning CGHR-10 CODs in the neonatal age group (under 28 days) by sex. PCCC values are sorted by the highest to lowest PCCC.}\label{fig-perf-sex-neo}
\end{figure}

\end{appendices}

%%===========================================================================================%%
%% If you are submitting to one of the Nature Portfolio journals, using the eJP submission   %%
%% system, please include the references within the manuscript file itself. You may do this  %%
%% by copying the reference list from your .bbl file, paste it into the main manuscript .tex %%
%% file, and delete the associated \verb+\bibliography+ commands.                            %%
%%===========================================================================================%%

\clearpage
\bibliography{bibliography}% common bib file
%% if required, the content of .bbl file can be included here once bbl is generated
%%\input sn-article.bbl


\end{document}
